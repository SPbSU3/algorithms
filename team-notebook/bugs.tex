\begin{itemize}
    \item \foreignlanguage{russian}{Общее}
        \begin{enumerate}
            \item \foreignlanguage{russian}{В задаче просят выводить массив int-ов посортированным, а я это не делаю.}
            
            \item \foreignlanguage{russian}{Бесконечность = максимальному возможному числу
(когда это приводит к нежелательному результату).}

            \item \foreignlanguage{russian}{НЕправильно: v.resize(q); forn (i, q) v.pb(...)\\ правильно: v.resize(q); forn (i, q) v[i] = ...}

            \item \foreignlanguage{russian}{Аккуратнее с глобальными переменными. В частности, не использовать общий used
при рекурсивном подсчёте чисел Гранди в стиле
tmr++; for (int x : sons) used[grundy(x)] = tmr;
(проблема в том, что used-ы могут перезаписаться, когда мы пойдём в ребёнка).}

        \end{enumerate}

    \item \foreignlanguage{russian}{Геометрия} 
        \begin{enumerate}
            \item \foreignlanguage{russian}{Когда смотришь, на какой угол надо повернуть прямую, чтобы она пересекла
            многоугольник, получается на самом деле два отрезка углов (так как поворот прямой
                на a in [0, pi) и a + pi есть одно и то же).}
            \item \foreignlanguage{russian}{Не писать проверку принадлежности точки многоугольники
            с площадями в случае даблов (хреново с точностью для точек около
            границы). Правильно с суммой углов.}
        \end{enumerate}

\end{itemize}

